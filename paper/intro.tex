\section{Introduction}
\label{s:intro}

Cloaking is a technique to deliver blatantly different content to users versus
censors on the web. 
Search engine is the main entry for user to get online today, and therefore,
attackers who want to promote illegal services or do phishing and malware
hosting is willing to have their page ranked high through search engine.
On the other hand, search engine develops algorithms to rank pages and penalize
these pages. In order to evade detection, attackers use cloaking to hide the
true nature of their website, conducting blackhat Search Engine Optimization (SEO).
Similarly, in Search Engine Marketing (SEM), advertisers pay search engine for
specific advertisements and get their page displayed on the first page of search
results. Attackers abuse this service to deliver non-compliant content to user.
The foundamental reason of SEO and SEM cloaking is that, search engine is just processing
information over the Internet, with no control over it, hence it is hard to
tell whether the copy of website that they see are actually what user sees.

In SEO and SEM, attackers use user agent, referer, IP etc based cloaking
techniques to avoid inspection. Among the cloaking techniques, IP cloaking is
very difficult to detect. 
There are two major challenges in cloaking detection in SEO and SEM.
First challenge is revealing the cloaked content. There are various
commercial tools, such as blackjack's, noIPfraud, and wpCloaker~\cite{intro:cloakware}
that are efficient in performing cloaking.
They provide different strategies of cloaking, including user agent, referer,
redirect, IP etc. Regarding IP cloaking, they provide services to periodically refresh the list of
crawlers' IP addresses. Therefore, revealing cloaked content is an ongoing war
between cloakers and censors.
Second challenge is differentiating
cloaking from naturally dynamic changes of the same page. Dynamic content is
allowed in SEO and SEM, however, blatantly deliver different, unrelated content
to search engine spider versus normal user is not allowed.

In order to address the first challenge, existing detection approaches pretend to be real user, e.g. set
referer, user agent, use multiple IPs or proxy to hide identity. But
attackers could identify them and incrementally blacklist IPs.
Intuitively, an alternate is to collect data from user side directly. However, it is hard
to collect sufficient data for comparison while protecting user privacy.
Regarding the second challenge, previous works go through the documents multiple times and compare page
difference to address the second challenge ~\cite{chellapilla2006improving, deng2013uncovering,
lin2009detection, wang2011cloak}.
There are mainly two shortcomings in previous approaches.
First,  they are slow, because pairwise comparison of documents is
computationally expensive and they usually require multiple passes of a
document.
Second, they are not suitable for crowdsourcing, because they may introduce
high overhead to user and violates user privacy. 

%To address the second challenge, ~\cite{wu2005cloaking} looks at the
%statistical features of a website, including words, tags, links. 
%~\cite{chellapilla2006improving} employs popular search words and high
%monetizable words to detect cloaking, say, detect cloaking in words that matter.
%~\cite{lin2009detection} looks at the tag sequence on a website.
%~\cite{wang2011cloak} employs text shingling, search snippet inclusion test in
%landing page and tag comparison to detect cloaking. However, these methods are
%either too naive, eg. look at statistics, or time-consuming and
%redundant~\cite{wang2011cloak}. Besides, since all these approaches are
%centralized detection, they can not be deployed to detect SEM cloaking, because
%the data collection phase, say, clicking search ads, actually results in click fraud.

This work employs simhash to address the two challenges.
Charikar's simhash ~\cite{charikar2002similarity} is a special signature of feature set. 
It is a class of Locality Sensitive Hash (LSH) that has been extensively used in
near duplicate in search engine.
It is an random projection based algorithm that maps high dimentional data to fixed bits while
maintaining the property that, hamming distance between the resulting bits is an
estimation of similarity between the original feature set.
%estimation of the Earth Mover Distance (EMD) between the original feature set.

%In our approach, we stand back and think about the dynamic nature of websites
%and try to figure out how can we measure or quantify the dynamics of websites.
%For example, does the change of a website follow some pattern? What is the
%average change? By answering these questions, we can then figure out the
%websites and then model them better. We first illustrate the dynamicness of a
%webpage and propose a way to model their change.
%Next, we train our model on manually label dataset to check some paramter to
%identify cloaking (one use case for website modeling). 
%Then, we employ our model to detect cloaking on advertisement cloaking dataset
%and identified \XXX{??} cloaking sites, showing prevalence of cloaking in both
%advertisements and search.
%
We propose Simhash-based Website Model(SWM) to model the dynamics of a
web page and use the outlier detection ability of SWM to do cloaking detection.
The basic idea is to generate two fuzzy hash corresponding to text and dom tree
for each website (fuzzy signature), and compare the differences between spider copy and user copy.
In order to model dynamics of a website, we retrieve spider copy multiple
times and leverage the fact that hamming distance is euclidean distance (
L1 norm) to learn patterns.
The proposed system achieves 97.1\% true positive rate with
0.3\% false positive rate.
%And much higher efficiency compared to past
%approaches. 

By applying the trained model on collected dataset, we detected 2600 cloaking samples in total.
Then, we manually classify and illustrate the incentives behind cloaking,
results show that the majority are doing blackhat SEO for traffic sale,
4.3\% of cloaking sites are malicious download or phishing sites. A breakdown of
the results are described in ~\autoref{s:measurement}.

With motivation of both search engine and user to combat cloaking, we propose a
novel model to collect simhash
%
%we establish incentives both search engine and users in cloaking detection.
%both user and search engine are interested in this.
%
from user side, with low overhead and privacy guarantee provided by RAPPOR.
By crowdsourcing the fuzzy signatures of what is actually shown to user, we
address both challenges in cloaking elegantly.

%In the new model, we can detect cloaking in SEM, because what we need to collect
%is url and statistics of simhash, which doesn't comply to click fraud. 

%Past approaches cannot handle advertisements, however, because of privacy issues
%and efficient hash technique. In comparison, our model is light-weight, easy to
%implement algorithm that can be used to fingerprint a website. And this
%fingerprint can be used to either compare against known blacklist, or send back
%to search engine for further analysis.
%


%According to ~\cite{lin2009detection}, many cloaking software packages support
%IP cloaking, and some even provide services to periodically refresh the list of
%crawlers’ IP addresses. Therefore, IP cloaking is hard to detect. To the
%author's knowledge, there is no known work that can solve IP
%cloaking.


%Security problems in cloaking, phishing site, suspicious site, non-compliant
%content, malware, dishonest behavior.
%
%Given a form to the user and tell him, you can get some money if you sigh up,
%but this form is not delivered to Google.

%Precision is 0.9610642439974043, out of 1541.
%
%This paper makes the following contributions:
%(1) Introduce simhash-based website model to represent the content and layout
%dynamics of a page.
%(2) Propose the idea and framework to detect cloaking and protect user through crowdsourcing,
%with low traffic overhead and privacy guarantee for user.
%(3) Use of simhash-based website model, in cloaking detection, with better FPR
%and TPR.
%(4) Detect cloaking in both SEO and SEM , showing the seriousness of cloaking
%problem in SEO and SEM.
%(5) Examine the cloaking strategies employed by attackers.
%

The remainder of this paper is structured as follows. Section
~\autoref{s:related-work} provides a
technical background on cloaking and simhash, and related work in cloaking
detection. Section ~\autoref{s:methodology} introduces methodlogies, including simhash-based website
model, cloaking detection algorithm and dataset collection.
Followed by evaluation of our model in Section ~\autoref{s:evaluation}.
Section ~\autoref{s:measurement} explains our detection result in SEO and SEM
and motivate the deployment.
Section ~\autoref{s:discussion} compares our work with past approaches and 
describes server-based and crowdsource-based
deployment, proposes the frameworks and corresponding pros and cons. Section
~\autoref{s:conclusion} concludes this work.

%More fascinating text. Features\endnote{Remember to use endnotes, not
%footnotes!} galore, plethora of promises.\\
